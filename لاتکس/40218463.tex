\documentclass{article}
\usepackage{xepersian}
\settextfont{XB Roya}
\title{کلاس کارگاه کامپیوتر}
\date{5 اردیبهشت}
\author{فرزاد ابراهیمی}
\begin{document}
\maketitle
در جلسه گذشته درمورد برنامه LaTEX صحبت شد. از کاربرد ها و رقبا و افزونه ها و قابلیت های ویژه این برنامه. این برنامه توسط دونالد کنوث ساخته شده که یکی از سرشناس های رشته علوم کامپیوتر و ریاضیات است. این برنامه یک سیستم حروف چینی بر پایه TEX است که قابلیت های بسیاری برای متن های زیاد و سخت به وجود آورده است. این برنامه از افزونه های زیادی پشتیبانی میکند و قابلیت های زیادی رو به برنامه اضافه کرده است. این برنامه از برنامه های رایگان هست,به همین دلیل دسترسی نویسنده ها و خبرنگار ها رو در نوشتن متن باز نگه داشته.
در انتهای جلسه به برنامه همچون Word که ساخته مایکروسافت است اشاره شد که مانند LaTEX یک برنامه حروف چینی است که جزئی از رقبای این برنامه به حساب می آید. از این برنامه اغلب برای نوشتن مقاله علمی, کتاب ها و اسناد ها استفاده میشود. LaTEX در برابر برنامه ای همچون Word شاید ساده و کلاسیک به نظر بیاید ولی دست کمی از آن ندارد و قدرت زیادی در نوشتن مقالات و اسناد دارد. یک افزونه مهم برا ما ایرانیا که توانسته است کار ما را راحت تر کند افزونه xepersian است که در این مقاله نیز استفاده شده است که حروف ها را راست چین و اعداد را به فارسی و به کل قابلیت تایپ و نوشتن فارسی را در اختیار ما قرار داده است.
\end{document}